\input luaotfload.sty

\font\hyten=file:GHEAMariamReg.otf at 12pt

Helo, {\bf world}{\it !}

\hyten
Armenian, Հայերեն

Մարկ Տուէնը — ամերիկական կատակաբան է։ Նա սիրւած հեղինակ է իւր երկրում եւ յարգանք է վայելում Եւրոպայում։ Նորա գրւածքները թարգմանւած են եւրոպական լեզուներով։

Ամուսինների վրդովմունքը բկացաւի պատճառով

(Պատմել է հեղինակին Նիւ֊Եօրքի քաղաքացի Մակ Ուիլեամսը, որին հեղինակը պատահել է երկաթուղու կառքում)։

Պատմածս վերաբերում է, այսպէս սկսեց Ուիլեամսը, այն ժամանակին, երբ կրուպ անւանւած սարսափելի եւ անբուժելի բկացաւը տարածուած էր մեր քաղաքումը եւ ահ էր գցել բոլոր մայրերի վերայ։ Այդ ժամանակն էր, որ ես իմ՝ կնոջ ուշադրութիւնը հրաւիրելով մեր փոքր աղջիկ Նելլու վերայ՝ ասացի․

— Ես, հոգեակս, եթէ քո տեղ լինէի, թոյլ չէի տալ, որ երեխան կրծի այս սոճի ճիպոտը։

— Էհ, ի՞նչ վնաս ունի, սիրելիս, պատասխանեց կինս եւ միեւնոյն ժամանակը պատրաստւեց պաշտպանելու երեխայի արածը, որովհետեւ կանայքը չեն կարող առանց հակառակելու ամենալաւ խորհուրդն անգամ ընդունել։ Իմ խօսքը պսակած կանանց մասին է։

— Բոլորը գիտեն, նկատեցի ես կնոջս, որ սոճին ամենից անսննդարար ծառն է, որ կրծում է երեխան։

Կինս, որ ձեռքը տարել էր ճիպոտը առնելու, յետ քաշեց իւր ձեռքը։ Լաւ երեւում էր, որ տիկինս ներքուստ կռւում է եւ վերջապէս ասաց․

— Դու, ի հարկէ, սիրելիս, այդ բանը աւելի լաւ իմանում կը լինես, կամ գոնէ երեւակայում ես, որ գիտես։ Բայց բոլոր բժիշկները ասում են, որ սոճի ծառի մէջ եղած հիւթը, տերպենտինը, շատ օգտակար է այն երեխաներին, որոնց մէջքը եւ երիկամունքը թոյլ են կազմւած։

— Հա՛, եթէ այդպէս է, ես սխալ եմ։ Դէ՛ ես չը գիտէի, որ Նելլու երիկամունքը եւ ողնաշարը վնասւած են եւ որ բժիշկը խորհուրդ է տւել…

— Ո՞վ ասաց, թէ աղջկանս մէջքը եւ երիկամունքը վնասւած են։

— Միթէ՞, հոգեա՛կս, այս րոպէին դո՛ւ չէիր ակնարկութիւն անողը։

— Ի՞նչ ես բաներ հնարում։ Ես այդ տեսակ ոչ մի ակնարկուիւն չեմ արել։

— Ինչպէ՞ս, աղաւնեակս։ Դեռ երկու րոպէ էլ չի անցել, որ դու ասացիր թէ…

— Փիլիսոփայութիւններ մի անիլ, խնդրեմ։ Ես ի՞նչ գիտեմ, ինչ ասացի։ Աղջկայ ճիպոտ կրծելուց ոչ մի վնաս չի կարող պատահել․ այդ ինքդ էլ լաւ գիտես դու։ Թող կրծի իրան համար․ ի՞նչ ես ուզում։

— Բաւական է, հոգեակս․ ես այժմ հասկացայ քո կարծիքը այդ բանի մասին եւ կը գնամ կը կարգադրեմ, որ հենց այսօր Նելլու համար պատրաստեն երկու կամ երեք սայլ լաւ սոճի փայտ։ Իմ երեխան չը պէտք է պակասութիւն զգայ, թէ ինչ է, ես…

— Ո՛ւհ, գնաս մէկ դու քո գործիդ եւ ինձ հանգիստ թողնես։ Կինդ մի խօսք չը պէտք է ասի, որ դու չբռնես նորանից։ Եւ յետոյ կը վիճես֊կը վիճես, մինչեւ որ ինքդ էլ չես իմանալ, թէ ինչից սկսւեց բոլոր խօսակցութիւնը։

— Լաւ, քո ասածը լինի․ բայց ախր հետեւողութիւն չկայ քո այն վերջին նկատողութեան մէջ, որով…

Բայց իմ՝ անկոտրում կինը դեռ չʼվերջացած խօսքիցս բռնկւեց եւ դուրս վազեց, հետը տանելով եւ աղջկաս։

Երեկոյեան, սեղանի վերայ նորից տեսայ կնոջս,, բայց սարսափած, սփրտնած, պատից աւելի սպիտակած երեսով։

— Գիտե՞ս ինչ, ա՛ մարդ, շտապեց նա ասելու, դարձեալ մի զոհ։ Փոքրիկ Գէորգը հիւանդացել է։

— Չը լինի՞ կրուպով։

— Այո կրուպով։

— Բայց մի՞թէ փրկելու յոյս չը կայ։

— Ոչ մի․ Տէր ամենակալ, ի՞նչ պէտք է լինի մեր վիճակը։

Այդ ժամանակ դայեակը բերաւ Նելլուն բարի գիշեր ասելու եւ մօր ետեւից գիշերուայ աղօթքը կրկնելու։ Աղօթքի ժամանակ՝ ինչպէս եղաւ երեխան հազաց։ Կինս սարսափից ընկաւ, կարծես կայծակահար եղած։ Բայց շուտով ոտքի թռաւ՝ երկիւղից առաջացած հոգսերին նւիրւելու։

Նա հրամայեց Նելլու մահճակալը մանկասենեակից հեռացնել մեր ննջարանը եւ գնաց ստուգելու հրամանի կատարումը։ Ինձ էլ քաշ տւեց իւր հետը։

Մեծ արագութեամբ ամեն բան կարգադրւած էր։ Կնոջս առանձնարանում դրւած էր մահճակալ դայեակի համար։ Բայց կինս յանկարծ մտաբերեց, որ այստեղ մենք շատ հեռացած կը լինենք միւս երեխայից եւ դժւար վիճակի մէջ կը լինենք, եթէ ցաւեր սկսւեն նորանում։ Խեղճ կին֊արմատս սարսափից կրկին գունատւեց։

Այն ժամանակ երեխայի եւ դայեակի մահճակալները տեղաւորեցինք հարեւան սենեակում։

Հազիւ գործը վերջացրել էինք, կինս մտաբերեց, որ Նելլին կարող է Գէորգից վարակւել։ Այդ նոր ծագած կասկածը այնպէս վրդովեցրեց նորան, որ մենք չը գիտէինք՝ էլ ինչպէս նորան հանգստացնենք։ Աներեւակայելի արագութեամբ դուրս հանեցինք մահճակալները մանկասենեակից։ Կինս այնպէս կատաղի կերպով շտապում էր, որ քիչ մնաց ջարդուփշուր անի Նելլու մահճակալը։

Մենք իջանք ստորին յարկը, բայց այնտեղ դայեակի համար չը կարողացանք տեղ յարմարեցնել, մինչդեռ կինս մեծ յոյս էր դրել՝ դայեակի փորձառութեան վերայ։ Ստիպւած՝ մենք կրկին տարանք բոլոր իրեղէնները մեր ննջարանը եւ վերջապէս փոթորիկից ցրիւ եկած եւ նորի բունը գտած թռչունների նման՝ ուրախացանք, երբ կարգադրութիւններս իրագործեցինք։

ԿԻնս անհամբեր վազեց մանկասենեակը տեսնելու՝ ինչ է այնտեղ կատարւում․ բայց շատ շուտ դարձաւ նոր սարսափով եւ հարցրեց ինձանից։

— Սա ի՞նչ է նշանակում, փոքրիկը ձիգ քնում է։

— Հոգեա՛կս, դու լաւ գիտես, որ մեր փոքրիկը միշտ լաւ է քնում։

— Գիտեմ, գիտեմ․ բայց այժմ նա ուրիշ կերպ է քնած։ Նա կարծես… կարծես… Այնպէս հանգիստ շունչ է քաշում… Տէր Աստուած, ես վախենում եմ…

— Դէ՛ հոգեակս, նա միշտ հանգիստ է քնում։

— Ա՛խ, այդ գիտեմ․ բայց հիմի այդ բանը ինձ կարծես վախեցնում է։ Նորան հսկող աղջիկը շատ է ջահիլ ու անփորձ։ Մարիամին ուղարկենք նորա մօտ․ ո՞վ գիտէ, ինչ կարող է պատահել այնտեղ։

— Այդ ասենք՝ լաւ ես մտածել․ բայց քեզ ով պէտք է օգնի, հարցրի ես։

— Ա՛յ, դու կարող ես ամեն բանում ինձ օգնել։ Այսպէս սարսափելի ժամանակ՝ ես ոչ ոքի ինքնուրոյն գործ չեմ հաւատալ․ ինքս պէտք է անեմ։

Ես կնոջս պատասխանեցի, որ ես անվայել կը համարեմ պառկել եւ բնել — նորա վզին գցելով մեր հիւանդ երեխայի խնամքը եւ հոգատարութիւնը ամբողջ տանջալի գիշերը։ Բայց կինս ինձ հանգստացրեց։ Պառաւ Մարիամը գնաց առաջւայ պէս մանկասենեակում մնալու։

Նելլին քնի մէջ երկու անգամ հազաց։

— Ա՛խ, ի՞նչ եղաւ բժիշկը։ Ա՛ մա՛րդ, սենեակումը շատ շոք է, անտանելի է։ Փակիր վառարանի պահակը։

Կատարեցի պատւէրը․ բայց նայելով ջերմաչափին, որ ցոյց էր տալիս միայն 16°, զարմացայ թէ ինչու է այդ աստիճանը շատ երեւացել հիւանդ երեխայի համար։

Այդ ժամանակ կառապանը քաղաքի հեռու ծայրից տուն դարձաւ, լուր բերելով, որ մեր բժիշկը ինքը հիւանդ է եւ պառկած է անկողնում։ Կինս նայեց ինձ վերայ մոլոր հայեացքով եւ յուսահատ ձայնով ասաց․

— Ես այստեղ ճակատագիրս եմ տեսնում։ Երեւի մեզ այդպէս է վիճակւած։ Նա առաջ երբէք, երբէք հիւանդ չէր լինում։ Երեւի Աստուծու օրէնքի դէմ ենք ապրել, ահա որտեղ է բանը։ Քանի հազար անգամ քեզ ասել եմ այդ մասին։ Այժմ պտուղները տեսնում ես։ Մեր երեխան էլ չի առողջանալ։ Երանի քեզ, որ այս դասը դու չես զգում։ Իսկ ես ինքս ինձ չեմ ներիլ երբէք։

Չը ցանկանալով ի՛հարկէ կնոջը վիրաւորելու եւ միայն անզգուշութիւնիցս ես պատասխանեցի, որ ես երբէք չեմ էլ մտածել, թէ մենք անազնիւ, անբարոյական կեանք ենք վարել։

— Ա՛ մարդ, ուզո՞ւմ ես այժմ էլ Աստուծու բարկութիւնը երեխիս գլխին թափես։

Կինս սկսեց լալ եւ յանկարծ բացականչեց։

— Բժիշկը գոնեա դեղեր ուղարկէ՛ր։

Ես պատասխանեցի՝

— Ի հարկէ։ Ահա դեղերը։ Ես սպասում էի յարմար րոպէի, որ քեզ հաղորդեմ։

— Դէ, որ այդպէս է, տուր այստեղ շուտով, դէհա։ Թէ չե՞ս հասկանում, որ ամեն մի վայրկեանը թանգ է։ Ի՞նչ եմ ասում․ ի՞նչ եմ անում դեղերը, երբ բժշկի ասելով՝ այս հիւանդութիւնը անբուժելի է։

Ես պատասխանեցի, որ քանի որ երեխան կենդանի է, յոյսներս չը պէտք է կտրենք։

— Յոյսնե՞րս, ճչաց կինս, հասկանում ես ի՞նչ ես խօսում, չʼծնած երեխայի չափ էլ… Սա ի՞նչ է նշանակում, դեղ֊ստոմսակի վերայ գրած է «ժամը մի անգամ մի թէյ գդալ»։ Ժամը՞ մի անգամ։ Կարծես թէ երեխայի կեանքը փրկելու համար մի տարի ժամանակ ունենք։ Օհ, շտապի՛ր, ա մարդ, դեղը բաց արա։ Դու մի հացի գդալ ածա․ դէհա, շուտ արա։

— Բայց, սիրելիս, յանկարծ հացի գդալից…

— Հոգիս մի հանիլ, Աստուած սիրես՝ մէկ է… Ա՛ռ, ա՛ռ, ընդունիր այս դեղը, իմ սիրունիկս, իմ քաղցրիկս, մի քիչ դառն կը լինի, բայց յետոյ Նելլիկս, մայրիկի աղջիկը, կʼառողջանաս։ Այ, այդպէս, ալ։ Ապրես։ Դէ հիմի գլուխդ դիր բարձիկին եւ քնիր, աչքերդ փակիր, փակիր, սիրունիկս… Օ՛հ, ես գիտեմ, նա մինչեւ էգուց չի ապրիր։ Գիտես ինչ (դարձաւ նա ինձ)․ ե՛կ դեղը տանք հացի գդալով ամեն մի կէս ժամից յետոյ։ Թէ ասում եմ, երեխային մի քիչ էլ բելադօն տանք, օկօնիտ տանք, կʼօգնեն։ Գնաս, ա՛ մարդ, բերես։ Բայց սպասիր, լաւ կը լինի՝ ես ինքս գնամ։ Դու այսպէս բաների գլուխ չʼունես։

Շուտով դորանից յետոյ մենք պառկեցինք քնելու՝ երեխայի մահճակալը մօտիկացնելով կնոջս բարձին։ Այս տար ու բերը ինձ լաւ յօգնեցրեց, այնպէս որ հազիւ պառկեցի՝ քունս սաստիկ տարաւ։ Բայց կինս զարթեցրեց։

— Սիրելի՛ս, ասաց նա, վառարանի պահակը հօ բաց չի՞։

— Ոչ։

— Ես էլ այդպէս էի կարծում։ Վերկաց, մէկ բաց արա՛․ սենեակումը ինչ որ շատ է ցուրտ։

Ես կատարեցի նորա խնդիրքը եւ կրկին քնեցի, բայց մի քանի ժամանակից յետոյյ՝ կինս ինձ նորից զարթնեցրեց։

— Որ երեխայի մահճակալը քո անկողնին մօտեցնես, լաւ կʼանես․ վառարանին մօտ կը լինի երեխան՝ կը տաքանայ։

Ես ուզում էի մահճակալը տեղափոխել, բայց ոտքս դիպաւ գորգին եւ երեխան զարթնեց։ Կինս սկսեց նորան քնացնել, իսկ ես մնջացի։ Կարճ ժամանակից յետոյ լսեցի կնոջս շշնչալը։

— Սիրելիս, սագի ճրագու է հարկաւոր, մենք անենք, բայց պէտք է բերող լինի։ Զանգը տուր, աղախինը գայ։

Քնաթաթախ վերկացայ եւ յանկարծ կոխ տւի կատուին, որ սաստիկ ճիչ հանեց։ Ես քացով տւի նորան, բայց ոտքս դիպաւ բազկաթոռին։

— Ա՛ մարդ, ի՞նչու ես լուսաւորում սենեակը․ գիտես, որ երեխան կարող է զարթնել։

— Մէկ չը տեսնե՞մ, ո՞ր տեղս եմ չջարդել կամ վիրաւորե՞լ։

— Դէ որ այդպէս է, հէնց բազկաթոռն էլ նաի՛ր… կոտրւած է լինելու անպատճառ։ Խե՛ղճ կատու՛ եթէ քացիդ նորան հասած լինէր…

— Կատուի դարդը մէկ թող է՛։ Այս բաները տեղի չէին ունենալ․ եթէ դու Մարիամին այստեղ պահած լինէիր, որ նա ինքը կատարէր այս նորան միայն վերաբերելի գործերը։

— Ամօ՛թ քեզ, դեռ տղամարդ ես․ ես կարծում էի, դու կը քաշւես այդպիսի բաներ ասել։ Ես վրդովւում եմ, երբ լսում եմ, որ դու ինձ ամենաչնչին գործողութիւններ չես կարողանում անել այսպիսի պարագաներում, երբ մեր երեխան…

— Լաւ, լաւ, լաւ․ բոլորը կʼանեմ ինչ ուզում ես։ Բայց զանգակ տալը զուր է, ամենքը քնած են։ Ո՞րտեղ է դրւած սագի ճրադուն։

Մանկասենեակի վառարանի վերայ է։ Այնտեղ ես գնում, ասա՛ Մարիամին, որ օգնի։

Ես բերի ճրագուն եւ պառկեցի քնելու․ բայց կինս ինձ հանգիստ չը թողեց։

— Գիտես, սիրելիս, որ ես քեզ նեղութիւն տալուց ինքս տանջւում եմ, բայց սենեակը էլի ցուրտ է, ճրագուն քսել չի կարելի։ Վառարանը մէկ վառես։ Փայտը դարսւած է, միայն լուցկին վառես ու կպցնես՝ բաւական է։

Վերկացայ տեղիցս, վառարանը վառեցի եւ վհատւած կերպով ընկայ աթոռի վերայ։

— Ինչո՞ւ ես աթոռի վերայ նստում, ա՛ մարդ, կը մրսես։ Մտի՛ր անկողինդ։

Երբ որ պառկում էի, կինս ասաց․

— Մի քիչ էլ սպասի՛ր․ Նելլին մի անգամ էլ դեղ ընդունի։

Դեղն էլ տւի։ Դեղը փախցնում էր երեխայի քունը։ Կինս հէնց որ տեսնում էր, որ երեխան զարթնել է, իսկոյն նորան մերկացնում էր ու մարմինը սագի ճրագու քսում։ Ես դարձեալ քնեցի եւ դարձեալ ինձ զարթեցրին։

— Սիրելիս, մի տեղից փչում է եւ սաստիկ է փչում։ Իսկ այս հիւանդութեան ժամանակ քամին շատ վնաս է։ Երեխայի մահճակալը մօտիկացրու կրակին։

Այս պատւէրը կատարելուց՝ ոտքս դարձեալ դիպաւ գորգին․ ես բարկացայ եւ զայրոյթից վեր առի գորգը վառարան գցելու․ բայց կինս անկողնից թռաւ եւ ձեռքիցս խլեց գորգը։ Մենք մի թեթեւ քնեցինք։ Փոքր քնելուց յետոյ՝ ես վերկացայ եւ կտաւատի սերմ դրի երեխայի կրծքին։

Որովհետեւ վառարանում կրակը ինքն իրան չի մնում՝ հանգչում է, ուստի ամեն մի քանի րոպէն վեր էի կենում եւ ուղղում էի փայտերը։ Այս հանգամանքից օգուտ քաղեց կինս եւ նա էլ քսան րոպէից յետոյ երեխային դեղ էր տալիս։ Ժամանակ֊ժամանակ, այն էլ բաւական ստէպ, փոխում էի կտաւատի սերմը եւ մանանեխ ու ուրիշ սպեղանի էի դնում երեխայի մարմնի ազատ մնացած տեղերում։ Այս բոլորը մի կերպ տանում էի․ բայց առաւօտեան դէմ՝ փայտերը բոլորովին այրւեցին եւ կինս ստիպեց, որ ես գնամ փայտանոցը եւ փայտի նոր պաշար բերեմ։ Ես չը համբերեցի եւ ասացի․

— Հոգեակս, փայտ բերելը շատ էլ հեշտ ու հաճելի չէ։ Երեխային կարելի է տաքացնել լաւ ծածկելով։ Փոխենք մարմնի վերայ դրած դեղերը եւ այնուհետեւ…

Կինս մի աղմուկ հանեց, որ, ի՞նչ արած, գնացի մի խտիտ փայտ բերի, որից յետոյ ընկայ անկողինս եւ այնպէս խռմփացրի, ինչպէս ընդունակ են ֆիզիքապէս եւ հոգեպէս շատ յոգնած մարդիկը։ Առաւօտը՝ լուսաբացին՝ ես զգացի մի խիստ հարւած, որ ինձ սթափեցրեց։ Կինս նայում էր ինձ վերայ մոլոր հայեացքով եւ վրդովմունքից շնչասպառ էր լինում։ Մի կերպ ուժը ժողովելով նա ասաց․

— Բոլորը վերջացաւ, բոլորը կորաւ։ Տե՛ս ի՛նչպէս երեխան քրտնած է։ Ի՞նչ անենք այժմ, Տէր Աստուած, Տէր Աստուած։

— Հեր օ՛րհնած, մի՞թէ կարելի է այդպէս վախեցնել։ Տէր Աստուած, չը գիտեմ՝ ինչ անեմ։ Որ այդպէս է, նորից ճրագու քսի՛ր ու դնենք երեխային քամի տեղ, որ…

— Օհ, ինչ խելառն ես։ Ամեն մի րոպէն այստեղ թանգ է։ Շուտով գնա բժշկի ետեւից։ Ինքդ գնա՛ — եւ նորան ասա, գլխին դիր վերջապէս, որ պէտք է գայ այստեղ ինքը՝ կենդանի թէ մեռած։

Ես վազեցի, հիւանդ բժշկին ոտքի կանգնացրի եւ հետս առի ու տուն եկայ։ Նա քննեց Նելլուն եւ յայտնեց, որ նա մեռնելու ամենեւին ցանկութիւն չունի։ Ես ի հարկէ շատ ուրախացայ, բայց կինս այնպէս բարկացաւ, կատաղեց, որ կարծես թէ բժիշկը նորան վիրաւորել էր իւր խօսքով։ Յետոյ նա ասաց, որ երեխայի հազի պատճառը նորա կոկորդը ընկած մի որ եւ է իրի աննշան գրգռումից է առաջ եկել։ Ես կարծում էի՝ թէ կինս բժշկի այդպիսի խօսքերի վրայ դուրս կʼանի նորան սենեակից։ Բժիշկը աւելացրեց յետոյ, որ եթէ երեխան մի անգամ սաստիկ հազայ, կոկորդը կարող է մաքրւել։ Եւ արդարեւ բժիշկը տւեց երեխային մի ինչ որ դեղ, որից նորա հազը սաստիկ բռնեց եւ կոկորդից դուրս թռաւ մի կտոր բարակ բան։

— Ձեր աղջիկը կրուպով հիւանդ չէ, յայտարարեց բժիշկը։ Կարելի է նա մի որ եւ է ճիւղի կտոր կամ ճիպոտ է կրծել, որից մի կտոր մնացել է կոկորդում։ Դորանից մի առանձին վնաս չի առաջանալ։

— Ես էլ այդ կարծիքի եմ, աւելացրի ես։ Բայց մի բան կայ, որ սոճում եղած տեպենտինը շատ օգտաւէտ է մանկական մի քանի հիւանդութիւնների ժամանակ։ Սակայն իմ կինը ինքը աւելի լաւ կերպով կը բացատրի ձեզ այդ բանը։

Կինս դատողութիւն անելու գլուխ չունէր։ Նա արհամարհանքով շուռ տւեց երեսը եւ սենեակից դուրս եկաւ։ Այդ ժամանակից մեր ընտանեկան կեանքում մէկ միջնադէպք պատահեց, որի մասին մենք աշխատում ենք երբէք խօսք չը բաց անել․ բայց եւ այդ ժամանակից մեր օրերը անցնում են ուրախ եւ անվրդով։

______________________

Պսակւածներից շատ սակաւներն են ծանօթ պարոն Ուիլեամսի քաշած տանջանքին, ուստի հեղինակը կարծում է, որ այս դէպքի նորութիւնը հետաքրքրական կը լինի բարեմիտ ընթերցողի համար։ 

\bye
