\font\hysix="GHEAGrapalat:mapping=tex-text" at 6pt
\font\hysixi="GHEAGrapalat/I:mapping=tex-text" at 6pt
\font\hysixb="GHEAGrapalat/B:mapping=tex-text" at 6pt

\font\hyeight="GHEAGrapalat:mapping=tex-text" at 8pt
\font\hyeighti="GHEAGrapalat/I:mapping=tex-text" at 8pt
\font\hyeightb="GHEAGrapalat/B:mapping=tex-text" at 8pt

\font\hyten="GHEAGrapalat:mapping=tex-text" at 10pt
\font\hyteni="GHEAGrapalat/I:mapping=tex-text" at 10pt
\font\hytenb="GHEAGrapalat/B:mapping=tex-text" at 10pt

\def\tenpoint{%
    \def\am{\fam0\hyten}%
    \textfont0=\hyten \scriptfont0=\hyeight \scriptscriptfont0=\hysix
    \textfont\itfam=\hyteni \scriptfont\itfam=\hyeighti 
      \scriptscriptfont\itfam=\hysixi \def\it{\fam\itfam\hyteni}%
    \textfont\bffam=\hytenb \scriptfont\bffam=\hyeightb
      \scriptscriptfont\bffam=\hysixb \def\bf{\fam\bffam\hytenb}%
    \normalbaselineskip=14pt
    \normalbaselines\am}

\font\symbol="DejaVu Sans Mono" at 10pt

\pdfpagewidth=105mm
\pdfpageheight=148.5mm

\hsize=75mm
\vsize=120mm

\hoffset=-10.4mm
\voffset=-10.4mm

\parindent=0pt
\nopagenumbers


\newcount\problemcount \problemcount=0
\def\problem{%
  \ifnum\problemcount>0\solution\fi
  \vfill\eject\tenpoint%
  \advance\problemcount by1\null}

\newdimen\freespace
\def\solution{\bigskip
{\symbol\offinterlineskip
  \freespace\vsize \advance\freespace-\pagetotal
  \special{color push rgb 0.8 0.8 0.8}
  \leftline{┌─┬─┬─┬─┬─┬─┬─┬─┬─┬─┬─┬─┬─┬─┬─┬─┬─┐}
  \loop
    \leftline{├─┼─┼─┼─┼─┼─┼─┼─┼─┼─┼─┼─┼─┼─┼─┼─┼─┤}
    \advance\freespace by -12pt
  \ifdim\freespace>60pt\repeat
  \leftline{└─┴─┴─┴─┴─┴─┴─┴─┴─┴─┴─┴─┴─┴─┴─┴─┴─┘}
  \special{color pop}}}

\tolerance=4000
\frenchspacing
\hyphenchar\hyten="058A % տողադարձի նշանը
\tenpoint
%\uselanguage{english}

\problem 
Կարի գործարանից առաջին օրը տարան 120 վերնաշապիկ, 
երկրորդ օրը՝ 92 վերնաշապիկ: Պահեստում մնաց 56 
վերնաշապիկ: Ընդա\-մենը քանի՞ վերնաշապիկ կար պահեստում:

\problem 
Տակառում կար 55 լիտր կաթ: Առաջին օրը տակառի մեջ 
լցրեցին 16 լիտր կաթ, իսկ երկրորդ օրը տակառից 
դատարկեցին 8 լիտր կաթ: Որքա՞ն կաթ մնաց տակառում:

\problem
Մարտի 15-ին գրախանութը ստացավ 62 գիրք: Նույն օրը 
վաճառեց 40 գիրք: Որքա՞ն գիրք մնաց այդ գրախանութում, 
եթե մարտի 14-ին կար 703 գիրք:

\problem 
Դպրոց բերեցին 56 կարմիր և 46 դեղին կակաչ: Ձևավորման 
համար օգտագործեցին 20 դեղին կակաչ: Ընդամենը քանի՞ 
կակաչ մնաց:

\problem
Մանկապարտեզի երեխաների համար բերե\-ցին 5 կգ խնձոր և 4 
կգ տանձ: Երեխաներին տվեցին 2 կգ խնձոր և 1 կգ տանձ: 
Որքա՞ն միրգ մնաց մանկապարտեզում:

\problem 
Երկու ծորակից լողավազան լցվեց 85 լ ջուր։ Երրորդ ծորակով 
լողավազանից դուրս թափ\-վեց 22 լ ջուր: Որքա՞ն ջուր մնաց 
լողավա\-զանի մեջ։

\problem 
4 տուփի մեջ կա 48 մատիտ: Որքա՞ն կլինի մա\-տիտների 
քանակը, եթե 2 տուփի մեջ ավելաց\-նենք 3-ական մատիտ:

\problem 
Շուկայում մայրիկը 2 կգ լոլիկի և 1 կգ վարունգի դիմաց վճարեց 
680 դրամ: 1 կգ դդմիկն էժան է 1 կգ վարունգից 150 դրամով: 
Որքա՞ն վճարեց մայրիկը 2 կգ լոլիկի և 1 կգ դդմիկի դիմաց:

\problem
Դպրոցի փոքր դահլիճում տեղավորեցին 8 շարք աթոռ, յուրաքանչյուր 
շարքում՝ 7 աթոռ: Ընդամենը քանի՞ աթոռ տեղավորվեց այդ դահլիճում:

\problem
Բաժակները տեղավորեցին 12 արկղի մեջ՝ յուրաքանչյուրում 10 
բաժակ: Ընդամենը քանի՞ բաժակ տեղավորեցին այդ արկղերում:

\problem
Վարսիկն իր հավաքած գումարը տեղա\-վորել էր մայրիկի նվիրած 
գեղեցիկ, փոքրիկ դրամա\-պանակում: Դպրոցական տարբեր պա\-րագաներ 
գնելու համար նա իր հավաքած գումարից վերցրեց 2000 դրամանոց 
2 թղթա\-դրամ և 100 դրամանոց 3 մետաղադրամ: Դրամապանակում 
մնաց 1500 դրամ: Որքա՞ն դրամ ուներ Վարսիկը։

\problem
Մայրիկը խանութից գնեց հաց, որի դիմաց վճարեց 250 դրամ, 
և մի տուփ սուրճ՝ 460 դրամով: Գնումներից հետո մայրիկի 
մոտ մնաց 1530 դրամ: Որքա՞ն գումար ուներ մայրիկը նախքան 
գնումներ կատարելը:

\problem
Երեք թվերի գումարը 80 է։ Առաջին թիվը երկրորդից փոքր 
է 5 անգամ: Երրորդն առա\-ջինից մեծ է 10 անգամ: Որ՞ն է 
երրորդ թիվը, եթե երկրորդ թիվը 25-ն է

\problem
Երեք թվերից երկրորդն առաջինից մեծ է 5 անգամ, իսկ 
երրորդից մեծ է 10 անգամ: Գտի՛ր առաջին թիվը, եթե 
երրորդ թիվը 15-ն է: 08

\problem
Տոնածառի մեկ խաղալիքն արժե 60 դրամ, իսկ բացիկը՝ 2 
անգամ ավելի։ Որքա՞ն պետք է վճարեն երեխաները տոնածառի 
3 խաղալիք և 3 բացիկ գնելու համար։

\problem
Արմանը եթե եռապատկի ընկերոջ մտապահած թիվը, ապա կստանա 
200-ից 40 միավորով մեծ թիվ: Ո՞ր թիվն է մտապահել Արմանի 
ընկերը:

\problem
Իմ մտապահած թվի քառապատիկը 10 անգամ մեծ է 20-ից: Ո՞ր 
թիվն եմ մտապահել։

\problem
Դասասենյակ բերեցին դպրոցական 15 նստարան: Եթե յուրաքանչյուր 
նստարանին նստի 2 աշակերտ, ապա 3 աշակերտ կմնա ոտքի վրա: 
Քանի՞ աշակերտ կա դասարանում:

\problem
Հարսանյաց սրահում կա 30 սեղան և 155 աթոռ: Քանի՞ աթոռ 
կավելանա, եթե յուրաքանչյուր սեղանի շուրջ դրվի 4 աթոռ:

\problem
Շրջազգեստ կարելու համար օգտագործեցին 3~մ կտոր: Քանի՞ 
մետր կտոր է անհրաժեշտ 4 նմանատիպ շրջազգեստ կարելու համար։

\problem
Թոռնիկների համար տատիկը գործեց ձեռնոցներ: Մեկ զույգ 
ձեռնոցի համար անհրաժեշտ է 35 մ կարմիր թել։ Քանի՞ մետր 
թել կօգտագործի տատիկը 5 զույգ ձեռնոց գործելու համար:

\problem
Խոհարարների համար կարեցին 7 գոգնոց: Յուրաքանչյուր 
գոգնոցի համար օգտագործեցին 2 մ գործվածք: Քանի՞ մետր 
գործվածք է անհրաժեշտ 7 գոգնոցի համար։

\problem
Հավասար տարողությամբ չորս տակառի մեջ լցրեցին 40 կաթ: 
Քանի՞ լիտր կաթ լցրեցին յուրաքանչյուր տակառի մեջ։

\bye
