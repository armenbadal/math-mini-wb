\input wbmac.tex

%% ՏԻՏՂՈՍԱԹԵՐԹ



%% ՍԿԻԶԲ

% I տասնյակ
\textproblem Երեք թվերից երկրորդն առաջինից մեծ է 5 անգամ,
իսկ երրորդից մեծ է 10 անգամ: Գտի՛ր առաջին թիվը, եթե երրորդ
թիվը 15-ն է:\answer{30}

\textproblem Եռանկյան կողմերից մեկը 14 սմ է։ Երկրորդ կողմը
6 սմ կարճ է այդ կողմից, իսկ երրորդ կողմը հավասար է առաջին
կողմին: Հաշվի՛ր եռանկյան պարագիծը:\answer{36}

\textproblem Կոշիկի արտադրամասը մեկ ամսում կարեց տղամարդու,
կնոջ և մանկան 500 զույգ կոշիկ: Տղամարդու կոշիկը 130 զույգ
է, իսկ կնոջ կոշիկը՝ 150 զույգ։ Քանի՞ զույգ մանկան կոշիկ
կարեցին:\answer{220}

\textproblem Պապիկը 60 տարեկան է: Թոռնիկը՝ Արամը, 10 անգամ
փոքր է պապիկից: Նրա եղբայրը 3 տարով մեծ է Արամից։ Քանի՞
տարեկան է Արամի եղբայրը:\answer{13}

\textproblem Տակառում կար 55 լիտր կաթ: Առաջին օրը տակառի
մեջ լցրեցին 16 լիտր կաթ, իսկ երկրորդ օրը տակառից դատարկեցին
8 լիտր կաթ: Որքա՞ն կաթ մնաց տակառում:\answer{63}

\textproblem Մարտի 15-ին գրախանութը ստացավ 62 գիրք: Նույն
օրը վաճառեց 40 գիրք: Որքա՞ն գիրք մնաց այդ գրախանութում,
եթե մարտի 14-ին կար 703 գիրք:\answer{725}

\textproblem A piece of cardboard is 2 mm thick. Suppose
it was folded in half, then folded in half again, and
folded in half once more. How thick is the folded piece
of cardboard?\answer{64}

\textproblem Hadley buys 7 boxes of markers with 8 in each box.
Delaney buys 6 boxes of markers with 12 markers in each box.
How many markers did Hadley and Delaney buy in all?
\answer{128}

\textproblem В одном альбоме 350 марок, что на  100 марок
больше, чем в другом альбоме. Сколько  всего марок в этих
альбомах?\answer{800}

\textproblem  Два опытных участка имеют одинаковую площадь.
Ширина первого участка 60 м, а ширина второго – 80 м. Найди
длину первого участка, если длина второго участка 150 м.
\answer{200}


% II տասնյակ
\textproblem Դպրոց բերեցին 56 կարմիր և 46 դեղին կակաչ:
Ձևավորման համար օգտագործեցին 20 դեղին կակաչ: Ընդամենը
քանի՞ կակաչ մնաց:\answer{82}

\textproblem Մանկապարտեզի երեխաների համար բերեցին 5 կգ
խնձոր և 4 կգ տանձ: Երեխաներին տվեցին 2 կգ խնձոր և 1 կգ
տանձ: Որքա՞ն միրգ մնաց մանկապարտեզում:\answer{6}

\textproblem Երկու ծորակից լողավազան լցվեց 85 լ ջուր։
Երրորդ ծորակով լողավազանից դուրս թափվեց 22 լ ջուր: Որքա՞ն
ջուր մնաց լողավազանի մեջ։\answer{63}

\textproblem 4 տուփի մեջ կա 48 մատիտ: Որքա՞ն կլինի
մատիտների քանակը, եթե 2 տուփի մեջ ավելացնենք 3-ական
մատիտ:\answer{54}

\textproblem Դպրոցի փոքր դահլիճում տեղավորեցին 8 շարք
աթոռ, յուրաքանչյուր շարքում՝ 7 աթոռ: Ընդամենը քանի՞ աթոռ
տեղավորվեց այդ դահլիճում:\answer{56}

\textproblem Բաժակները տեղավորեցին 12 արկղի մեջ՝
յուրաքանչյուրում 10 բաժակ: Ընդամենը քանի՞ բաժակ
տեղավորեցին այդ արկղերում:\answer{120}

\textproblem Twelve coworkers go out for lunch together
and order three pizzas. Each pizza is cut into eight
slices. If each person gets the same number of slices,
how many slices will each person get?\answer{2}

\textproblem A goat drinks about 5 liters of water every
day. If a rancher has a water tank that holds 60 liters
of water for his six pet goats, how many days will the
water last?\answer{2}

\textproblem С одного участка школьники собрали 161 кг
свеклы, а с другого 289 кг Всю свеклу они разложили в
ящики по 15 кг в каждый ящик. Сколько потребовалось
ящиков для свеклы?\answer{30}

\textproblem В магазин привезли 14 кг огурцов и
28 кг помидоров. За день продали 24 кг овощей.
Сколько килограммов овощей  осталось в магазине?\answer{f}


% III տասնյակ
\textproblem Մայրիկը խանութից գնեց հաց, որի դիմաց
վճարեց 250 դրամ, և մի տուփ սուրճ՝ 460 դրամով:
Գնումներից հետո մայրիկի մոտ մնաց 1530 դրամ: Որքա՞ն
գումար ուներ մայրիկը նախքան գնումներ կատարելը:\answer{2240}

\textproblem Երեք թվերի գումարը 80 է։ Առաջին թիվը
երկրորդից փոքր է 5 անգամ: Երրորդն առաջինից մեծ է 10
անգամ: Որ՞ն է երրորդ թիվը, եթե երկրորդ թիվը 25-ն է:\answer{55}

\textproblem Երեք թվերի գումարը 660 է: Առաջին գումարելին
200 է, իսկ երկրորդ գումարելին 5 անգամ փոքր է առաջին
գումարելիից: Ինչի՞ է հավասար երրորդ գումարելին:\answer{260}

\textproblem Պահեստից խանութ առաջին օրը ուղարկեցին
20 կգ գազար, կիլոգրամը 100 դրամ արժողությամբ։ Երկրորդ
օրը ուղարկեցին 30 կգ գազար, իսկ երրորդ օրը՝ 34 կգ
գազար: Որքա՞ն պետք է վճարեր խանութը պահեստին:\answer{8400}

\textproblem Ցանկապատի 1 մ${}^2$ ներկելու համար վարպետին
վճարում են 500 դրամ։ Որքա՞ն կստանա վարպետը՝ 24 օր աշխատելով,
եթե նա մեկ օրում ներկում է 12 մ${}^2$ մակերես:\answer{144000}

\textproblem 500 կգ բեռը կայարանից երկաթուղով պետք է
ուղարկեին 3 օրում: Առաջին օրն ուղարկեցին 100 կգ, որը
2 անգամ քիչ է երկրորդ օրվա ուղարկածից: Քանի՞ կիլոգրամ
բեռ ուղարկեցին երրորդ օրը։\answer{200}

\textproblem Wendy has 5 pairs of pants and 7
shirts. How many different outfits can she make?\answer{35}

\textproblem Lisa was assigned 48 pages to read
for English class. She has finished $4/3$ of
the assignment. How many more pages must she read?\answer{12}

\textproblem Велосипедист проехал 60 км за 5 ч. За
какое время он проехал бы этот путь, если бы увеличил
скорость на 3 км/ч?\answer{4}

\textproblem В каждой коробке 24 конфеты. На первый стол
поставили 3 коробки, на второй 4 коробки. На сколько конфет
больше на втором столе, чем на первом?\answer{24}


% IV տասնյակ
\textproblem Խառատը մեկ ժամում պատրաստում է 300 դետալ:
Օրական նա պետք է պատրաստի 1800 դետալ: 4 ժամ աշխատելուց
հետո քանի՞ դետալ նրան կմնա պատրաստելու։\answer{600}

\textproblem 90 ձմերուկի դիմաց վճարեցին 4500 դրամ ավելի,
քան նույն գնով 70 ձմերուկի դիմաց: Որքա՞ն պետք է վճարել 50
ձմերուկի դիմաց:\answer{15000}


\textproblem Erica was 132 cm tall when she was 9 years
old. In the next year, she grew 6 cm, and the next year
2 cm less than the previous year. How tall was she at
the age of 11?\answer{144}


% V տասնյակ

% VI տասնյակ

% VII տասնյակ

% VIII տասնյակ

% IX տասնյակ

% X տասնյակ

\bye
