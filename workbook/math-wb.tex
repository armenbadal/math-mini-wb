\input wbmac.tex

%% ՏԻՏՂՈՍԱԹԵՐԹ



\hyphenation{ձեռ-նոց-ներ պա-րա-գա-ներ արկ-ղից պատ-րաս-տե-ին
յու-րա-քանչ-յու-րում շր-ջա-զգեստ գործ-վածք ու-ղար-կե-ին}


%% ՍԿԻԶԲ

\textproblem Կարի գործարանից առաջին օրը տարան 120
վերնաշապիկ, երկրորդ օրը՝ 92 վերնաշապիկ: Պահեստում
մնաց 56 վերնաշապիկ: Ընդամենը քանի՞ վերնաշապիկ կար
պահեստում:\answer{34}

\textproblem Հաշվել արտահայտության արժեքը.
$$72840 : 50 + 1290\times 4$$\answer{8}

\textproblem A piece of cardboard is 2 mm thick. Suppose
it was folded in half, then folded in half again, and
folded in half once more. How thick is the folded piece
of cardboard?\answer{64}

\textproblem С одного участка школьники собрали 161 кг
свеклы, а с другого 289 кг Всю свеклу они разложили в
ящики по 15 кг в каждый ящик. Сколько потребовалось
ящиков для свеклы?\answer{a}

\problem Նկարել սլաքները նշված ժամերի համար։

\def\analogclock{{\XeTeXpicfile figures-1.png width 2.7cm}}

\bigskip
\centerline{\vbox{\halign{\hss#\hss&\quad#\quad&\hss#\hss\cr
\analogclock&&\analogclock\cr
13:15&&20:00\cr
\noalign{\bigskip}
\analogclock&&\analogclock\cr
10:55&&18:20\cr
\noalign{\bigskip}
\analogclock&&\analogclock\cr
00:05&&3:45\cr}}}


\textproblem В магазин привезли 14 кг огурцов и
28 кг помидоров. За день продали 24 кг овощей.
Сколько килограммов овощей  осталось в магазине?\answer{f}

\problem 
Տակառում կար 55 լիտր կաթ: Առաջին օրը տակառի մեջ 
լցրեցին 16 լիտր կաթ, իսկ երկրորդ օրը տակառից 
դատարկեցին 8 լիտր կաթ: Որքա՞ն կաթ մնաց տակառում:

\problem
Մարտի 15-ին գրախանութը ստացավ 62 գիրք: Նույն օրը 
վաճառեց 40 գիրք: Որքա՞ն գիրք մնաց այդ գրախանութում, 
եթե մարտի 14-ին կար 703 գիրք:

\problem Hadley buys 7 boxes of markers with 8 in each box.
Delaney buys 6 boxes of markers with 12 markers in each box.
How many markers did Hadley and Delaney buy in all?
\answer{128}

\problem 
Դպրոց բերեցին 56 կարմիր և 46 դեղին կակաչ: Ձևավորման 
համար օգտագործեցին 20 դեղին կակաչ: Ընդամենը քանի՞ 
կակաչ մնաց:

\problem
Մանկապարտեզի երեխաների համար բերե\-ցին 5 կգ խնձոր և 4 
կգ տանձ: Երեխաներին տվեցին 2 կգ խնձոր և 1 կգ տանձ: 
Որքա՞ն միրգ մնաց մանկապարտեզում:

\problem 
Երկու ծորակից լողավազան լցվեց 85 լ ջուր։ Երրորդ ծորակով 
լողավազանից դուրս թափ\-վեց 22 լ ջուր: Որքա՞ն ջուր մնաց 
լողավա\-զանի մեջ։

\problem 
4 տուփի մեջ կա 48 մատիտ: Որքա՞ն կլինի մա\-տիտների 
քանակը, եթե 2 տուփի մեջ ավելաց\-նենք 3-ական մատիտ:

\problem 
Շուկայում մայրիկը 2 կգ լոլիկի և 1 կգ վարունգի դիմաց վճարեց 
680 դրամ: 1 կգ դդմիկն էժան է 1 կգ վարունգից 150 դրամով: 
Որքա՞ն վճարեց մայրիկը 2 կգ լոլիկի և 1 կգ դդմիկի դիմաց:

\problem
Դպրոցի փոքր դահլիճում տեղավորեցին 8 շարք աթոռ, յուրաքանչյուր 
շարքում՝ 7 աթոռ: Ընդամենը քանի՞ աթոռ տեղավորվեց այդ դահլիճում:

\problem
Բաժակները տեղավորեցին 12 արկղի մեջ՝ յուրաքանչյուրում 10 
բաժակ: Ընդամենը քանի՞ բաժակ տեղավորեցին այդ արկղերում:

\problem
Վարսիկն իր հավաքած գումարը տեղա\-վորել էր մայրիկի նվիրած 
գեղեցիկ, փոքրիկ դրամա\-պանակում: Դպրոցական տարբեր պա\-րագաներ 
գնելու համար նա իր հավաքած գումարից վերցրեց 2000 դրամանոց 
2 թղթա\-դրամ և 100 դրամանոց 3 մետաղադրամ: Դրամապանակում 
մնաց 1500 դրամ: Որքա՞ն դրամ ուներ Վարսիկը։

\problem
Twelve coworkers go out for lunch together and order three 
pizzas. Each pizza is cut into eight slices. If each person 
gets the same number of slices, how many slices will each 
person get?

\problem
Մայրիկը խանութից գնեց հաց, որի դիմաց վճարեց 250 դրամ, 
և մի տուփ սուրճ՝ 460 դրամով: Գնումներից հետո մայրիկի 
մոտ մնաց 1530 դրամ: Որքա՞ն գումար ուներ մայրիկը նախքան 
գնումներ կատարելը:

\problem
Երեք թվերի գումարը 80 է։ Առաջին թիվը երկրորդից փոքր 
է 5 անգամ: Երրորդն առա\-ջինից մեծ է 10 անգամ: Որ՞ն է 
երրորդ թիվը, եթե երկրորդ թիվը 25-ն է

\problem
Երեք թվերից երկրորդն առաջինից մեծ է 5 անգամ, իսկ 
երրորդից մեծ է 10 անգամ: Գտի՛ր առաջին թիվը, եթե 
երրորդ թիվը 15-ն է:

\problem
Տոնածառի մեկ խաղալիքն արժե 60 դրամ, իսկ բացիկը՝ 2 
անգամ ավելի։ Որքա՞ն պետք է վճարեն երեխաները տոնածառի 
3 խաղալիք և 3 բացիկ գնելու համար։

\problem
Արմանը եթե եռապատկի ընկերոջ մտապահած թիվը, ապա կստանա 
200-ից 40 միավորով մեծ թիվ: Ո՞ր թիվն է մտապահել Արմանի 
ընկերը:

\problem
Իմ մտապահած թվի քառապատիկը 10 անգամ մեծ է 20-ից: Ո՞ր 
թիվն եմ մտապահել։

\problem
Դասասենյակ բերեցին դպրոցական 15 նստարան: Եթե յուրաքանչյուր 
նստարանին նստի 2 աշակերտ, ապա 3 աշակերտ կմնա ոտքի վրա: 
Քանի՞ աշակերտ կա դասարանում:

\problem
Հարսանյաց սրահում կա 30 սեղան և 155 աթոռ: Քանի՞ աթոռ 
կավելանա, եթե յուրաքանչյուր սեղանի շուրջ դրվի 4 աթոռ:

\problem
A goat drinks about 5 liters of water every day. If a 
rancher has a water tank that holds 60 liters of water 
for his six pet goats, how many days will the water last?

\problem
Շրջազգեստ կարելու համար օգտագործեցին 3~մ կտոր: Քանի՞ 
մետր կտոր է անհրաժեշտ 4 նմանատիպ շրջազգեստ կարելու համար։

\problem
Թոռնիկների համար տատիկը գործեց ձեռնոցներ: Մեկ զույգ 
ձեռնոցի համար անհրաժեշտ է 35 մ կարմիր թել։ Քանի՞ մետր 
թել կօգտագործի տատիկը 5 զույգ ձեռնոց գործելու համար:

\problem
Խոհարարների համար կարեցին 7 գոգնոց: Յուրաքանչյուր 
գոգնոցի համար օգտագործեցին 2 մ գործվածք: Քանի՞ մետր 
գործվածք է անհրաժեշտ 7 գոգնոցի համար։

\problem
Հավասար տարողությամբ չորս տակառի մեջ լցրեցին 40 կաթ: 
Քանի՞ լիտր կաթ լցրեցին յուրաքանչյուր տակառի մեջ։

\problem
Արկղերը լցրեցին խնձորով։ Եթե առաջին արկղից, որտեղ կա 120 
հատ խնձոր, 15 խնձոր լցնեն երկրորդ արկղի մեջ և 25 խնձոր 
երրորդ արկղի մեջ, ապա արկղերում խնձորների քանակները 
կհավասարվեն: Քանի՞ հատ խնձոր կա երկրորդ և երրորդ արկղում:

\problem
Մեծ եղբայրն ունի 890 դրամ: Եթե նա փոքր եղբորը տա 100 դրամ, 
իսկ փոքր եղբայրն էլ մեծ եղբորը տա 55 դրամ, ապա նրանք կունենան 
հավասար քանակով դրամ: Որքա՞ն դրամ ունի փոքր եղբայրը։

\problem
Տակառներից մեկում կա 55 լ գինի: Եթե այդ տակառից 12 լ 
գինի լցնենք երկրորդ տակառի մեջ, իսկ երկրորդից՝ 8 լ գինի 
առաջին տակառի մեջ, ապա տակառներում գինիների քանակությունները 
կհավասարվեն: Քանի՞ լիտր գինի կա երկրորդ տակառում:

\problem
Դպրոցի գրադարան առաջին օրը բերեցին 23 հատ գիրք, իսկ երկրորդ 
օրը՝ 8 հատ գրքով պակաս: Քանի՞ հատ գիրք բերեցին դպրոցի գրադարան 
այդ երկու օրում:

\problem
Թոռնիկները, պապիկին օգնելով, առաջին օրը փորեցին 32 մ հող, իսկ 
երկրորդ օրը՝ 6 մետր ավելի: Քանի՞ մետր հող փորեցին նրանք երկու օրում:

\problem
Լեռնագոգ գյուղի դպրոցում սովորում է 425 աշակերտ: Նրանք 83 
աշակերտով ավելի են Հացիկ գյուղի դպրոցի աշակերտներից: Քանի՞ 
աշակերտ ունի Հացիկ գյուղի դպրոցը։

\problem
Քույրիկս գնեց 6-ական սև և գունավոր մատիտներ: Սև մատիտը 20 
դրամով էժան է գունավոր մատիտից: Քույրիկս գունավոր մատիտների 
դիմաց վճարել էր 240 դրամ: Հաշվի՛ր, թե քույրիկս որքա՞ն է վճարել 
սև մատիտների դիմաց:

\problem
Քառակուսու պարագիծն 80 մմ է։ Հաշվի՛ր այն հավասարա-կողմ եռանկյան 
պարագիծը, որի մի կողմը քառակուսու կողմից մեծ է 5 մմ։

\problem
Երեք թվերի գումարն 87 է: Առաջին թիվը 27 է, որը 3 անգամ մեծ է 
երկրորդ թվից: Գտի՛ր երրորդ թիվը:

\problem
Երեք օրում մանկական նկարների ցուցահանդես հաճախեց 240 մարդ: Առաջին 
օրը հաճախեց 102 մարդ, որը 3 անգամ շատ է երկրորդ օրվա հաճախողներից: 
Քանի՞ մարդ հաճախեց երրորդ օրը։

\problem
Երեք թվերի գումարը 660 է: Առաջին գումարելին 200 է, իսկ երկրորդ 
գումարելին 5 անգամ փոքր է աաջին գումարելիից:: Ինչի՞ է հավասար 
երրորդ գումարելին:

\problem
Ճամբարում երեխաները պետք է ծաղիկ պատրաստեին: Առաջին խումբը 
պատրաստեց 12 ծաղիկ: Երկրորդ խումբը 2 անգամ շատ ծաղիկ պատրաստեց 
առաջին խմբից, իսկ երրորդ խումբը՝ 8-ով շատ առաջին խմբից: Քանի՞ 
ծաղիկ պատրաստեցին այդ խմբերը միասին:

\problem
Բերքահավաքին մասնակցեցին նաև դպրոցականների երեք խումբ: Առաջին 
խումբը հավաքեց 17 կգ բերք, երկրորդը՝ 2 անգամ շատ առաջինից: Երրորդ 
խումբը հավաքեց 11 կգ ավելի առաջին խմբից: Որքա՞ն բերք հավաքեցին 
այդ խմբերը միասին:

\problem
500 կգ բեռը կայարանից երկաթուղով պետք է ուղարկեին 3 օրում: Առաջին 
օրն ուղարկեցին 100 կգ, որը 2 անգամ քիչ է երկրորդ օրվա ուղարկածից: 
Քանի՞ կիլոգրամ բեռ ուղարկեցին երրորդ օրը։

\problem
Խանութում կա 5 պարկ ալյուր՝ յուրաքանչյուրում 90 կգ: Հաշվի՛ր, թե 
քանի՞ կիլոգրամ ալյուր կա այդ խանութում:

\problem
Արշավի գնալու համար աշակերտները գնեցին 3 արկղ պահածո, յուրաքանչյուրում՝ 
12 հատ: Քանի՞ տուփ պահածո գնեցին աշակերտները:

\problem
Շուկայից մայրիկը գնեց 500 գ թթու, որի դիմաց վճարեց 300 դրամ: Նույն 
թթվից գնեց տատիկների համար՝ 1 կգ և 2 կգ: Որքա՞ն պետք է վճարեն 
տատիկները թթուների համար։

\problem
Պահեստից խանութ առաջին օրը ուղարկեցին 20 կգ գազար, կիլոգրամը 100 
դրամ արժողությամբ։ Երկրորդ օրը ուղարկեցին 30 կգ գազար, իսկ երրորդ 
օրը՝ 34 կգ գազար: Որքա՞ն պետք է վճարեր խանութը պահեստին:

\problem
Կոշիկի արտադրամասը մեկ ամսում կարեց տղամարդու, կնոջ և մանկան 500 
զույգ կոշիկ: Տղամարդու կոշիկը 130 զույգ է, իսկ կնոջ կոշիկը՝ 150 
զույգ։ Քանի՞ զույգ մանկան կոշիկ կարեցին:

\problem
Տորթ ժելեի համար հարկավոր է 500 գ միրգ, 50 գ ժելե և շաքարավազ։ 
Որքա՞ն շաքարավազ է հարկավոր, որ ստացվի 700 գ տորթ-ժելե:

\problem
Եռանկյան կողմերից մեկը 14 սմ է։ Երկրորդ կողմը 6 սմ կարճ է այդ 
կողմից, իսկ երրորդ կողմը հավասար է առաջին կողմին: Հաշվի՛ր եռանկյան 
պարագիծը:

\problem
Պապիկը 60 տարեկան է: Թոռնիկը՝ Արամը, 10 անգամ փոքր է պապիկից: 
Նրա եղբայրը 3 տարով մեծ է Արամից։ Քանի՞ տարեկան է Արամի եղբայրը:


\bye
